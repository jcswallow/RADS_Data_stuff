\documentclass[a4paper,10pt]{article}
\usepackage[utf8]{inputenc}

%opening
\title{Test Git Hub Document}
\author{Joel C. Swallow}

\begin{document}

\maketitle

\begin{abstract}
Here is some things about how to use git hub.
\end{abstract}

\section{From Kristian's Linux Lesson No.2}

\subsection{Login Details}
user: jcswallow
email: j**480@******
password: ********

\subsection{Begining}
\begin{itemize}
 \item Tell git to track a directory: \texttt{git init} (initializes git in that directory).
 \item \texttt{git status} - gives current status: verry usefull.
 \item In the directory we have a \texttt{.gitignore} file (.=invisible) in which we add all the file extensions (file types) we dont want to add.
 \item To add a file ready to comitt: \texttt{got add \textit{FileName}}.
 \item to add everything: \texttt{git add *}
 \item NOTE: hard to undo after a comitt!
 \item To undo a git add for a file \textit{FileName}: \texttt{git reset \textit{FileName}}
 \item To undo a git add in general: \texttt{git reset}. 
\end{itemize}

\subsection{Versioning}


\end{document}
